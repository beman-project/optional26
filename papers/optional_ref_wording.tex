\documentclass[a4paper,10pt,oneside,openany,final,article]{memoir}
%% Common header for WG21 proposals ? mainly taken from C++ standard draft source
%%

%%--------------------------------------------------
%% basics
% \documentclass[a4paper,11pt,oneside,openany,final,article]{memoir}

\usepackage[american]
           {babel}        % needed for iso dates
\usepackage[iso,american]
           {isodate}      % use iso format for dates
\usepackage[final]
           {listings}     % code listings
\usepackage{longtable}    % auto-breaking tables
\usepackage{ltcaption}    % fix captions for long tables
\usepackage{relsize}      % provide relative font size changes
\usepackage{textcomp}     % provide \text{l,r}angle
\usepackage{underscore}   % remove special status of '_' in ordinary text
\usepackage{parskip}      % handle non-indented paragraphs "properly"
\usepackage{array}        % new column definitions for tables
\usepackage[normalem]{ulem}
\usepackage{enumitem}
\usepackage{color}        % define colors for strikeouts and underlines
\usepackage{amsmath}      % additional math symbols
\usepackage{mathrsfs}     % mathscr font
\usepackage[final]{microtype}
\usepackage[splitindex,original]{imakeidx}
\usepackage{multicol}
\usepackage{lmodern}
\usepackage{xcolor}
\usepackage[T1]{fontenc}
\usepackage{graphicx}
\usepackage{hyperref}
% \usepackage[pdftex,
%             bookmarks=true,
%             bookmarksnumbered=true,
%             pdfpagelabels=true,
%             pdfpagemode=UseOutlines,
%             pdfstartview=FitH,
%             linktocpage=true,
%             colorlinks=true,
%             plainpages=false,
%             allcolors={blue},
%             allbordercolors={white}
%             ]{hyperref}

% \usepackage[pdftex,
%   pdfauthor={Steve Downey},
%   pdftitle={dXXXpX},
%   pdfkeywords={},
%   pdfsubject={},
%   pdfcreator={Emacs 28.1.50 (Org mode 9.5.2)},
%   pdflang={English}]{hyperref}

\usepackage{memhfixc}     % fix interactions between hyperref and memoir
\usepackage{expl3}
\usepackage{xparse}
\usepackage{xstring}

\usepackage{url}  % urls in ref.bib
\usepackage{tabularx}  % don't use the C++ standard's fancy tables, they come with captions!

\special{pdf:minorversion 5} %set minorversion
\special{pdf:compresslevel 9} %set minorversion
\special{pdf:objcompresslevel 9} %set minorversion

% \pdfminorversion=5
% \pdfcompresslevel=9
% \pdfobjcompresslevel=2

\renewcommand\RSsmallest{5.5pt}  % smallest font size for relsize

\input{stdtex/layout}
\input{stdtex/styles}
\input{stdtex/macros}
\input{stdtex/tables}

\usepackage{amssymb,graphicx,stackengine,xcolor}
\protected\def\ucr{\scalebox{1}{\stackinset{c}{}{c}{-.2pt}{%
      \textcolor{white}{\sffamily\bfseries\small ?}}{%
      \rotatebox{45}{$\blacksquare$}}}}

\newenvironment{wording}{
  \chapterstyle{cppstd}
  \newcounter{scratchChapter}\setcounter{scratchChapter}{\value{chapter}}
  \settocdepth{part}
  \renewcommand\thesection{\ucr.\arabic{section}}
  \renewcommand\thesubsection{\makebox{$\ucr$}.\makebox{$\ucr$}.\arabic{subsection}}
} {
  \setcounter{chapter}{\value{scratchChapter}}
  \settocdepth{section}

}


% %% --------------------------------------------------
% %% fix interaction between hyperref and other
% %% commands
% \pdfstringdefDisableCommands{\def\smaller#1{#1}}
% \pdfstringdefDisableCommands{\def\textbf#1{#1}}
% \pdfstringdefDisableCommands{\def\raisebox#1{}}
% \pdfstringdefDisableCommands{\def\hspace#1{}}



%%--------------------------------------------------
%% add special hyphenation rules
\hyphenation{tem-plate ex-am-ple in-put-it-er-a-tor name-space name-spaces non-zero}

%%--------------------------------------------------
%% turn off all ligatures inside \texttt
\DisableLigatures{encoding = T1, family = tt*}

%% --------------------------------------------------
%% configuration
\input{stdtex/config}

\settocdepth{chapter}
\usepackage{minted}
\usepackage{fontspec}
\setromanfont{Source Serif Pro}
\setsansfont{Source Sans Pro}
% \setmonofont{Source Code Pro}

\begin{document}
\title{std::optional<T\&>}
\author{
  Steve Downey \small<\href{mailto:sdowney@gmail.com}{sdowney@gmail.com}> \\
  Peter Sommerlad \small<\href{mailto:peter.cpp@sommerlad.ch}{peter.cpp@sommerlad.ch}> \\
}
\date{} %unused. Type date explicitly below.
\maketitle

\begin{flushright}
  \begin{tabular}{ll}
    Document \#: & P2988R1 \\
    Date: & \today \\
    Project: & Programming Language C++ \\
    Audience: & LEWG
  \end{tabular}
\end{flushright}

\begin{abstract}
  We propose to fix a hole intentionally left in \tcode{std::optional} ---

  An optional over a reference such that the post condition on assignment is independent of the engaged state, always producing a rebound reference, and assigning a \tcode{U} to a \tcode{T} is disallowed by \tcode{static_assert} if a \tcode{U} can not be bound to a \tcode{T\&}.
\end{abstract}

\tableofcontents*

\chapter*{Changes Since Last Version}

\begin{itemize}
\item \textbf{Changes since R0},
  \begin{itemize}
  \item Wording update
  \end{itemize}
\end{itemize}

\chapter{Comparison table}
\section{Using a raw pointer result for an element search function}

This is the convention the C++ core guidelines suggest, to use a raw pointer for representing optional non-owning references.
However, there is a user-required check against `nullptr`, no type safety meaning no safety against mis-interpreting such a raw pointer, for example by using pointer arithmetic on it.

\begin{tabular}{ lr }
  \begin{minipage}[t]{0.45\columnwidth}
    \begin{minted}[fontsize=\small]{c++}
      Cat* cat = find_cat("Fido");
      if (cat!=nullptr) { return doit(*cat); }


    \end{minted}
  \end{minipage}
  &
    \begin{minipage}[t]{0.45\columnwidth}
      \begin{minted}[fontsize=\small]{c++}
        std::optional<Cat&> cat = find_cat("Fido");
        return cat.and_then(doit);

      \end{minted}
    \end{minipage}
\end{tabular}

\section{returning result of an element search function via a (smart) pointer}

The disadvantage here is that \tcode{std::experimental::observer_ptr<T>} is both non-standard and not well named, therefore this example uses \tcode{shared_ptr} that would have the advantage of avoiding dangling through potential lifetime extension.
However, on the downside is still the explicit checks against the \tcode{nullptr} on the client side, failing so risks undefined behavior.

  \begin{tabular}{ lr }
  \begin{minipage}[t]{0.45\columnwidth}
    \begin{minted}[fontsize =\small]{c++}
std::shared_ptr<Cat> cat = find_cat("Fido");
if (cat != nullptr) {/* ... */}

    \end{minted}
  \end{minipage}
  &
    \begin{minipage}[t]{0.45\columnwidth}
      \begin{minted}[fontsize=\small]{c++}
std::optional<Cat&> cat = find_cat("Fido");
cat.and_then([](Cat& thecat){/* ... */}

      \end{minted}
    \end{minipage}
  \end{tabular}
  \section{returning result of an element search function via an iterator}

  This might be the obvious choice, for example, for associative containers, especially since their iterator stability guarantees.
  However, returning such an iterator will leak the underlying container type as well necessarily requires one to know the sentinel of the container to check for the not-found case.

  \begin{tabular}{ lr }
  \begin{minipage}[t]{0.45\columnwidth}
    \begin{minted}[fontsize=\small]{c++}
std::map<std::string, Cat>::iterator cat
        = find_cat("Fido");
if (cat != theunderlyingmap.end()){/* ... */}

    \end{minted}
  \end{minipage}
  &
    \begin{minipage}[t]{0.45\columnwidth}
      \begin{minted}[fontsize=\small]{c++}
std::optional<Cat&> cat
        = find_cat("Fido");
cat.and_then([](Cat& thecat){/* ... */}

      \end{minted}
    \end{minipage}
  \end{tabular}

  \section{Using an optional<T*> as a substitute for optional<T\&>}

This approach adds another level of indirection and requires two checks to take a definite action.

  \begin{tabular}{ lr }
  \begin{minipage}[t]{0.45\columnwidth}
    \begin{minted}[fontsize=\small]{c++}
//Mutable optional
std::optional<Cat*> c = find_cat("Fido");
if (c) {
    if (*c) {
      *c.value() = Cat("Fynn", color::orange);
    }
}

    \end{minted}
  \end{minipage}
  &
    \begin{minipage}[t]{0.45\columnwidth}
      \begin{minted}[fontsize=\small]{c++}
std::optional<Cat&> c = find_cat("Fido");
if (c) {
  *c = Cat("Fynn", color::orange);
}

//or

o.transform([](Cat& c){
  c = Cat("Fynn", color::orange);
});
        \end{minted}
      \end{minipage}
\end{tabular}

\chapter{Motivation}
Other than the standard library's implementation of optional, optionals holding references are common. The desire for such a feature is well understood, and many optional types in commonly used libraries provide it, with the semantics proposed here.
One standard library implementation already provides an implementation of \tcode{std::optional<T\&>} but disables its use, because the standard forbids it.

The research in JeanHeyd Meneide's _References for Standard Library Vocabulary Types - an optional case study._ \cite{P1683R0} shows conclusively that rebind semantics are the only safe semantic as assign through on engaged is too bug-prone. Implementations that attempt assign-through are abandoned. The standard library should follow existing practice and supply an \tcode{optional<T\&>} that rebinds on assignment.

Additional background reading on \tcode{optional<T\&>} can be found in JeanHeyd Meneide's article _To Bind and Loose a Reference_ \cite{REFBIND}.

In freestanding environments or for safety-critical libraries, an optional type over references is important to implement containers, that otherwise as the standard library either would cause undefined behavior when accessing an non-available element, throw an exception, or silently create the element. Returning a plain pointer for such an optional reference, as the core guidelines suggest, is a non-type-safe solution and doesn't protect in any way from accessing an non-existing element by a \tcode{nullptr} de-reference. In addition, the monadic APIs of \tcode{std::optional} makes is especially attractive by streamlining client code receiving such an optional reference, in contrast to a pointer that requires an explicit nullptr check and de-reference.

There is a principled reason not to provide a partial specialization over \tcode{T\&} as the semantics are in some ways subtly different than the primary template. Assignment may have side-effects not present in the primary, which has pure value semantics. However, I argue this is misleading, as reference semantics often has side-effects. The proposed semantic is similar to what an \tcode{optional<std::reference_wrapper<T>>} provides, with much greater usability.

There are well motivated suggestions that perhaps instead of an \tcode{optional<T\&>} there should be an \tcode{optional_ref<T>} that is an independent primary template. This proposal rejects that, because we need a policy over all sum types as to how reference semantics should work, as optional is a variant over T and monostate. That the library sum type can not express the same range of types as the product type, tuple, is an increasing problem as we add more types logically equivalent to a variant. The template types \tcode{optional} and \tcode{expected} should behave as extensions of \tcode{variant<T, monostate>} and \tcode{variant<T, E>}, or we lose the ability to reason about generic types.

That we can't guarantee from \tcode{std::tuple<Args...>} (product type) that \tcode{std::variant<Args...>} (sum type) is valid, is a problem, and one that reflection can't solve. A language sum type could, but we need agreement on the semantics.

The semantics of a variant with a reference are as if it holds the address of the referent when referring to that referent. All other semantics are worse. Not being able to express a variant<T\&> is inconsistent, hostile, and strictly worse than disallowing it.

Thus, we expect future papers to propose \tcode{std::expected<T\&,E>} and \tcode{std::variant} with the ability to hold references.
The latter can be used as an iteration type over \tcode{std::tuple} elements.


\chapter{Design}

The design is straightforward. The \tcode{optional<T\&>} holds a pointer to the underlying object of type \tcode{T}, or \tcode{nullptr} if the optional is disengaged. The implementation is simple, especially with C++20 and up techniques, using concept constraints. As the held pointer is a primitive regular type with reference semantics, many operations can be defaulted and are \tcode{noexcept} by nature. See \cite{Downey_smd_optional_optional_T} and \cite{rawgithu58:online}. The \tcode{optional<T\&>} implementation is less than 200 lines of code, much of it the monadic functions with identical textual implementations with different signatures and different overloads being called.

In place construction is not supported as it would just be a way of providing immediate life-time issues.

\chapter{Shallow vs Deep const}

There is some implementation divergence in optionals about deep const for \tcode{optional<T\&>}. That is, can the referred to \tcode{int} be modified through a \tcode{const optional<int\&>}. Does \tcode{operator->()} return an \tcode{int*} or a \tcode{const int*}, and does \tcode{operator*()} return an \tcode{int\&} or a \tcode{const int\&}. I believe it is overall more defensible if the \tcode{const} is shallow as it would be for a \tcode{struct ref {int * p;}} where the constness of the struct ref does not affect if the p pointer can be written through. This is consistent with the rebinding behavior being proposed.

Where deeper constness is desired, \tcode{optional<const T\&>} would prevent non const access to the underlying object.


\chapter{Proposal}


Add a reference specialization for the std::optional template.

\chapter{Wording}


\begin{wording}


  \rSec2[optional.optional]{Class template \tcode{optional}}

  \rSec3[optional.optional.general]{General}

  \indexlibraryglobal{optional}%
  \indexlibrarymember{value_type}{optional}%
  \begin{addedblock}
  \begin{codeblock}
    namespace std {
      template <class T>
      class optional<T&> {
        public:
        using value_type = T&;

        // \ref{optional.ctor}, constructors
        constexpr optional() noexcept;
        constexpr optional(nullopt_t) noexcept;
        constexpr optional(const optional&);
        constexpr optional(optional&&) noexcept(@\seebelow@);
        template<class U = T>
        constexpr explicit(@\seebelow@) optional(U&&);
        template<class U>
        constexpr explicit(@\seebelow@) optional(const optional<U>&);
        template<class U>
        constexpr explicit(@\seebelow@) optional(optional<U>&&);

        // \ref{optional.dtor}, destructor
        constexpr ~optional();

        // \ref{optional.assign}, assignment
        constexpr optional& operator=(nullopt_t) noexcept;
        constexpr optional& operator=(const optional&);
        constexpr optional& operator=(optional&&) noexcept(@\seebelow@);
        template<class U = T> constexpr optional& operator=(U&&);
        template<class U> constexpr optional& operator=(const optional<U>&);
        template<class U> constexpr optional& operator=(optional<U>&&);
        template<class... Args> constexpr T& emplace(Args&&...);
        template<class U, class... Args> constexpr T& emplace(initializer_list<U>, Args&&...);

        // \ref{optional.swap}, swap
        constexpr void swap(optional&) noexcept(@\seebelow@);

        // \ref{optional.observe}, observers
        constexpr const T* operator->() const noexcept;
        constexpr T* operator->() noexcept;
        constexpr const T& operator*() const & noexcept;
        constexpr T& operator*() & noexcept;
        constexpr T&& operator*() && noexcept;
        constexpr const T&& operator*() const && noexcept;
        constexpr explicit operator bool() const noexcept;
        constexpr bool has_value() const noexcept;
        constexpr const T& value() const &;                                 // freestanding-deleted
        constexpr T& value() &;                                             // freestanding-deleted
        constexpr T&& value() &&;                                           // freestanding-deleted
        constexpr const T&& value() const &&;                               // freestanding-deleted
        template<class U> constexpr T value_or(U&&) const &;
        template<class U> constexpr T value_or(U&&) &&;

        // \ref{optional.monadic}, monadic operations
        template<class F> constexpr auto and_then(F&& f) &;
        template<class F> constexpr auto and_then(F&& f) &&;
        template<class F> constexpr auto and_then(F&& f) const &;
        template<class F> constexpr auto and_then(F&& f) const &&;
        template<class F> constexpr auto transform(F&& f) &;
        template<class F> constexpr auto transform(F&& f) &&;
        template<class F> constexpr auto transform(F&& f) const &;
        template<class F> constexpr auto transform(F&& f) const &&;
        template<class F> constexpr optional or_else(F&& f) &&;
        template<class F> constexpr optional or_else(F&& f) const &;

        // \ref{optional.mod}, modifiers
        constexpr void reset() noexcept;

        private:
        T *val;         // \expos
      };
    }
  \end{codeblock}

  \pnum
  Any instance of \tcode{optional<T\&>} at any given time either refers to a value or does not refer to a value.
  When an instance of \tcode{optional<T\&>} \defnx{refers to a value}{refers to a value!\idxcode{optional}},
  it means that an object of type \tcode{T}, referred to as the optional object's \defnx{referred to value}{referred to value!\idxcode{optional}},
  is pointed to from the storage of the optional object.
  When an object of type \tcode{optional<T\&>} is contextually converted to \tcode{bool},
  the conversion returns \tcode{true} if the object refers to a value;
  otherwise the conversion returns \tcode{false}.

  \pnum
  When an \tcode{optional<T\&>} object refers to a value,
  member \tcode{val} points to the referred to object.
  \end{addedblock}

  \rSec3[optional.ctor]{Constructors}

  \pnum
  The exposition-only variable template \exposid{converts-from-any-cvref}
  is used by some constructors for \tcode{optional}.
  \begin{codeblock}
    template<class T, class W>
    constexpr bool @\exposid{converts-from-any-cvref}@ =  // \expos
    disjunction_v<is_constructible<T, W&>, is_convertible<W&, T>,
    is_constructible<T, W>, is_convertible<W, T>,
    is_constructible<T, const W&>, is_convertible<const W&, T>,
    is_constructible<T, const W>, is_convertible<const W, T>>;
  \end{codeblock}

  \indexlibraryctor{optional}%
  \begin{itemdecl}
    constexpr optional() noexcept;
    constexpr optional(nullopt_t) noexcept;
  \end{itemdecl}

  \begin{itemdescr}
    \pnum
    \ensures
    \tcode{*this} does not refer to a value.

    \pnum
    \remarks
    For every object type \tcode{T} these constructors are constexpr constructors\iref{dcl.constexpr}.
  \end{itemdescr}

  \indexlibraryctor{optional}%
  \begin{itemdecl}
    constexpr optional(const optional& rhs) noexecpt = default;
  \end{itemdecl}

  \begin{itemdescr}
    \pnum
    \effects
    If \tcode{rhs} refers to a value, direct-non-list-initializes the referred to value
    with \tcode{rhs.val}.

    \pnum
    \ensures
    \tcode{rhs.has_value() == this->has_value()}.

    \pnum
    \throws
    Nothing

  \end{itemdescr}

  \indexlibraryctor{optional}%
  \begin{itemdecl}
    constexpr optional(optional&& rhs) noexcept = default;
  \end{itemdecl}

  \begin{itemdescr}
    \pnum

    \pnum
    \effects
    If \tcode{rhs} refers to a value, direct-non-list-initializes the referred to value
    with \tcode{rhs.val}.
    \tcode{rhs.has_value()} is unchanged.

    \pnum
    \ensures
    \tcode{rhs.has_value() == this->has_value()}.

  \end{itemdescr}



  \indexlibraryctor{optional}%
  \begin{itemdecl}
    template<class U = T> constexpr explicit(@\seebelow@) optional(U&& v);
  \end{itemdecl}

  \begin{itemdescr}
    \pnum
    \constraints
    \begin{itemize}
    \item \tcode{is_same_v<remove_cvref_t<U>, optional>} is \tcode{false}, and
    \item if \tcode{T} is \cv{} \tcode{bool},
      \tcode{remove_cvref_t<U>} is not a specialization of \tcode{optional}.
    \end{itemize}

    \pnum
    \mandates
    \begin{itemize}
    \item \tcode{is_constructible_v<std::add_lvalue_reference_t<T>, U>} is \tcode{true}
    \item \tcode{is_lvalue_reference<U>::value>} is \tcode{true}
    \end{itemize}

    \pnum
    \effects
    Direct-non-list-initializes the referred to value with \tcode{std::addressof(v)}.

    \pnum
    \ensures
    \tcode{*this} refers to a value.

  \end{itemdescr}

  \indexlibraryctor{optional}%
  \begin{itemdecl}
    template<class U> constexpr explicit(@\seebelow@) optional(const optional<U>& rhs);
  \end{itemdecl}

  \begin{itemdescr}
    \pnum
    \constraints
    \begin{itemize}
    \item \tcode{is_same_v<remove_cvref_t<U>, optional>} is \tcode{false}, and
    \item if \tcode{T} is \cv{} \tcode{bool},
      \tcode{remove_cvref_t<U>} is not a specialization of \tcode{optional}.
    \end{itemize}

    \pnum
    \mandates
    \begin{itemize}
    \item \tcode{is_constructible_v<std::add_lvalue_reference_t<T>, U>} is \tcode{true}
    \item \tcode{is_lvalue_reference<U>::value>} is \tcode{true}
    \end{itemize}

    \pnum
    \effects
    If \tcode{rhs} refers to a value,
    direct-non-list-initializes the referred to value with \tcode{addressof(rhs.value()))}.

    \pnum
    \ensures
    \tcode{rhs.has_value() == this->has_value()}.
  \end{itemdescr}


  \rSec3[optional.dtor]{Destructor}

  \indexlibrarydtor{optional}%
  \begin{itemdecl}
    constexpr ~optional() = default;
  \end{itemdecl}

  \rSec3[optional.assign]{Assignment}

  \indexlibrarymember{operator=}{optional}%
  \begin{itemdecl}
    optional& operator=(nullopt_t) noexcept;
  \end{itemdecl}

  \begin{itemdescr}
    \pnum
    \ensures
    \tcode{*this} does not contain a value.

    \pnum
    \returns
    \tcode{*this}.
  \end{itemdescr}

  \indexlibrarymember{operator=}{optional}%
  \begin{itemdecl}
    constexpr optional<T>& operator=(const optional& rhs) noexcept = default;
  \end{itemdecl}

  \begin{itemdescr}
    \pnum
    \ensures
    \tcode{rhs.has_value() == this->has_value()}.

    \pnum
    \returns
    \tcode{*this}.

  \end{itemdescr}

  \indexlibrarymember{operator=}{optional}%
  \begin{itemdecl}
    constexpr optional& operator=(optional&& rhs) noexcept = default;
  \end{itemdecl}

  \begin{itemdescr}
    \pnum
    \ensures
    \tcode{rhs.has_value() == this->has_value()}.

    \pnum
    \returns
    \tcode{*this}.
  \end{itemdescr}

  \indexlibrarymember{operator=}{optional}%
  \begin{itemdecl}
    template<class U = T> constexpr optional<T>& operator=(U&& v);
  \end{itemdecl}

  \begin{itemdescr}
    \pnum
    \constraints
    \begin{itemize}
    \item \tcode{is_same_v<remove_cvref_t<U>, optional>} is \tcode{false}, and
    \item if \tcode{T} is \cv{} \tcode{bool},
      \tcode{remove_cvref_t<U>} is not a specialization of \tcode{optional}.
    \end{itemize}

    \pnum
    \mandates
    \begin{itemize}
    \item \tcode{is_constructible_v<std::add_lvalue_reference_t<T>, U>} is \tcode{true}
    \item \tcode{is_lvalue_reference<U>::value>} is \tcode{true}
    \end{itemize}

    \pnum
    \effects
    Assigns the referred to value with \tcode{std::addressof(v)}.

    \pnum
    \ensures
    \tcode{*this} refers to a value.

  \end{itemdescr}

  \indexlibrarymember{operator=}{optional}%
  \begin{itemdecl}
    template<class U> constexpr optional<T>& operator=(const optional<U>& rhs);
  \end{itemdecl}

  \begin{itemdescr}
    \pnum
    \mandates
    \begin{itemize}
    \item \tcode{is_constructible_v<std::add_lvalue_reference_t<T>, U>} is \tcode{true}
    \item \tcode{is_lvalue_reference<U>::value>} is \tcode{true}
    \end{itemize}

    \pnum
    \ensures
    \tcode{rhs.has_value() == this->has_value()}.

    \pnum
    \returns
    \tcode{*this}.

  \end{itemdescr}


  \indexlibrarymember{emplace}{optional}%
  \begin{itemdecl}
    template <class U = T> optional& emplace(U&& u) noexcept;
  \end{itemdecl}

  \begin{itemdescr}
    \constraints
    \begin{itemize}
    \item \tcode{is_same_v<remove_cvref_t<U>, optional>} is \tcode{false}, and
    \item if \tcode{T} is \cv{} \tcode{bool},
      \tcode{remove_cvref_t<U>} is not a specialization of \tcode{optional}.
    \end{itemize}
    \pnum
    \effects
    Assigns \tcode{*this} \tcode{std::forward<U>(u)}

    \pnum
    \ensures
    \tcode{*this} refers to a value.

  \end{itemdescr}

  \rSec3[optional.swap]{Swap}

  \indexlibrarymember{swap}{optional}%
  \begin{itemdecl}
    constexpr void swap(optional& rhs) noexcept;
  \end{itemdecl}

  \begin{itemdescr}

    \pnum
    \effects
    *this and *rhs will refer to each others initial referred to objects.

  \end{itemdescr}


  \rSec3[optional.observe]{Observers}

  \indexlibrarymember{operator->}{optional}%
  \begin{itemdecl}
    constexpr T* operator->() const noexcept;
  \end{itemdecl}

  \begin{itemdescr}
    \pnum
    \returns
    \tcode{val}.

    \pnum
    \remarks
    These functions are constexpr functions.
  \end{itemdescr}

  \indexlibrarymember{operator*}{optional}%
  \begin{itemdecl}
    constexpr T&  operator*() const& noexcept;
    constexpr T&& operator*() const&& noexcept;
  \end{itemdecl}

  \begin{itemdescr}
    \pnum
    \expects
    \tcode{*this} refers to a value.

    \pnum
    \returns
    \tcode{*val}.

    \pnum
    \remarks
    These functions are constexpr functions.
  \end{itemdescr}

  \indexlibrarymember{operator bool}{optional}%
  \begin{itemdecl}
    constexpr explicit operator bool() const noexcept;
  \end{itemdecl}

  \begin{itemdescr}
    \pnum
    \returns
    \tcode{true} if and only if \tcode{*this} refers to a value.

    \pnum
    \remarks
    This function is a constexpr function.
  \end{itemdescr}

  \indexlibrarymember{has_value}{optional}%
  \begin{itemdecl}
    constexpr bool has_value() const noexcept;
  \end{itemdecl}

  \begin{itemdescr}
    \pnum
    \returns
    \tcode{true} if and only if \tcode{*this} refers to a value.

    \pnum
    \remarks
    This function is a constexpr function.
  \end{itemdescr}

  \indexlibrarymember{value}{optional}%
  \begin{itemdecl}
    constexpr const T& value() const &;
    constexpr T& value() &;
  \end{itemdecl}

  \begin{itemdescr}
    \pnum
    \effects
    Equivalent to:
    \begin{codeblock}
      if (has_value())
          return *value_;
      throw bad_optional_access();
    \end{codeblock}
  \end{itemdescr}

  \indexlibrarymember{value}{optional}%
  \begin{itemdecl}
    constexpr T&& value() &&;
    constexpr const T&& value() const &&;
  \end{itemdecl}

  \begin{itemdescr}

    \pnum
    \effects
    Equivalent to:
    \begin{codeblock}
      if (has_value())
          return *value_;
      throw bad_optional_access();
    \end{codeblock}
  \end{itemdescr}

  \indexlibrarymember{value_or}{optional}%
  \begin{itemdecl}
    template<class U> constexpr T value_or(U&& v) const &;
  \end{itemdecl}

  \begin{itemdescr}
    \pnum
    \mandates
    \tcode{is_copy_constructible_v<T> \&\& is_convertible_v<U\&\&, T>} is \tcode{true}.

    \pnum
    \effects
    Equivalent to:
    \begin{codeblock}
      return has_value() ? value() : static_cast<T>(std::forward<U>(u));
    \end{codeblock}
  \end{itemdescr}

  \indexlibrarymember{value_or}{optional}%
  \begin{itemdecl}
    template<class U> constexpr T value_or(U&& v) &&;
  \end{itemdecl}

  \begin{itemdescr}
    \pnum
    \mandates
    \tcode{is_move_constructible_v<T> \&\& is_convertible_v<U\&\&, T>} is \tcode{true}.

    \pnum
    \effects
    Equivalent to:
    \begin{codeblock}
      return has_value() ? value() : static_cast<T>(std::forward<U>(u));
    \end{codeblock}
  \end{itemdescr}

  \rSec3[optional.monadic]{Monadic operations}

  \indexlibrarymember{and_then}{optional}
  \begin{itemdecl}
    template<class F> constexpr auto and_then(F&& f) &;
    template<class F> constexpr auto and_then(F&& f) const &;
  \end{itemdecl}

  \begin{itemdescr}
    \pnum
    Let \tcode{U} be \tcode{invoke_result_t<F, T\&>}.

    \pnum
    \mandates
    \tcode{remove_cvref_t<U>} is a specialization of \tcode{optional}.

    \pnum
    \effects
    Equivalent to:
    \begin{codeblock}
      return has_value() ? std::invoke(std::forward<F>(f), value()) : result(nullopt);
    \end{codeblock}
  \end{itemdescr}

  \indexlibrarymember{and_then}{optional}
  \begin{itemdecl}
    template<class F> constexpr auto and_then(F&& f) &&;
    template<class F> constexpr auto and_then(F&& f) const &&;
  \end{itemdecl}

  \begin{itemdescr}
    \pnum
    Let \tcode{U} be \tcode{invoke_result_t<F, T\&>}.

    \pnum
    \mandates
    \tcode{remove_cvref_t<U>} is a specialization of \tcode{optional}.

    \pnum
    \effects
    Equivalent to:
    \begin{codeblock}
      return has_value() ? std::invoke(std::forward<F>(f), value()) : result(nullopt);
    \end{codeblock}
  \end{itemdescr}

  \indexlibrarymember{transform}{optional}
  \begin{itemdecl}
    template<class F> constexpr auto transform(F&& f) &;
    template<class F> constexpr auto transform(F&& f) const &;
  \end{itemdecl}

  \begin{itemdescr}
    \pnum
    Let \tcode{U} be \tcode{remove_cv_t<invoke_result_t<F, T\&>>}.

    \pnum
    \mandates
    \tcode{U} is a non-array object type
    other than \tcode{in_place_t} or \tcode{nullopt_t}.
    The declaration
    \begin{codeblock}
      U u(invoke(std::forward<F>(f), *val));
    \end{codeblock}
    is well-formed for some invented variable \tcode{u}.
    \begin{note}
      There is no requirement that \tcode{U} is movable\iref{dcl.init.general}.
    \end{note}

    \pnum
    \returns
    If \tcode{*this} refers to a value, an \tcode{optional<U>} object
    whose refered to value is the result of
    \tcode{invoke(std::forward<F>(f), *val)};
    otherwise, \tcode{optional<U>()}.
  \end{itemdescr}

  \indexlibrarymember{transform}{optional}
  \begin{itemdecl}
    template<class F> constexpr auto transform(F&& f) &&;
    template<class F> constexpr auto transform(F&& f) const &&;
  \end{itemdecl}

  \begin{itemdescr}
    \pnum
    Let \tcode{U} be
    \tcode{remove_cv_t<invoke_result_t<F, T\&>>}.

    \pnum
    \mandates
    \tcode{U} is a non-array object type
    other than \tcode{in_place_t} or \tcode{nullopt_t}.
    The declaration
    \begin{codeblock}
      U u(invoke(std::forward<F>(f), std::move(*val)));
    \end{codeblock}
    is well-formed for some invented variable \tcode{u}.
    \begin{note}
      There is no requirement that \tcode{U} is movable\iref{dcl.init.general}.
    \end{note}

    \pnum
    \returns
    If \tcode{*this} refers to a value, an \tcode{optional<U>} object
    whose refered to value is the result of
    \tcode{invoke(std::forward<F>(f), std::move(*val))};
    otherwise, \tcode{optional<U>()}.
  \end{itemdescr}

  \indexlibrarymember{or_else}{optional}
  \begin{itemdecl}
    template<class F> constexpr optional or_else(F&& f) const &;
  \end{itemdecl}

  \begin{itemdescr}
    \pnum
    \constraints
    \tcode{F} models \tcode{\libconcept{invocable}<>} and

    \pnum
    \mandates
    \tcode{is_same_v<remove_cvref_t<invoke_result_t<F>>, optional>} is \tcode{true}.

    \pnum
    \effects
    Equivalent to:
    \begin{codeblock}
      if (*this) {
        return *this;
      } else {
        return std::forward<F>(f)();
      }
    \end{codeblock}
  \end{itemdescr}

  \indexlibrarymember{or_else}{optional}
  \begin{itemdecl}
    template<class F> constexpr optional or_else(F&& f) &&;
  \end{itemdecl}

  \begin{itemdescr}
    \pnum
    \constraints
    \tcode{F} models \tcode{\libconcept{invocable}<>} and

    \pnum
    \mandates
    \tcode{is_same_v<remove_cvref_t<invoke_result_t<F>>, optional>} is \tcode{true}.

    \pnum
    \effects
    Equivalent to:
    \begin{codeblock}
      if (*this) {
        return std::move(*this);
      } else {
        return std::forward<F>(f)();
      }
    \end{codeblock}
  \end{itemdescr}

  \rSec3[optional.mod]{Modifiers}

  \indexlibrarymember{reset}{optional}%
  \begin{itemdecl}
    constexpr void reset() noexcept;
  \end{itemdecl}

  \begin{itemdescr}

    \pnum
    \ensures
    \tcode{*this} does not refer to a value.
  \end{itemdescr}



\end{wording}

\chapter{Impact on the standard}

A pure library extension, affecting no other parts of the library or language.

The proposed changes are relative to the current working draft~\cite{N4910}.

\chapter*{Document history}

\begin{itemize}
\item \textbf{Changes since R0}
  \begin{itemize}
  \item Wording Updates
  \end{itemize}
\end{itemize}

\renewcommand{\bibname}{References}
\bibliographystyle{abstract}
\bibliography{wg21,mybiblio}

\end{document}
