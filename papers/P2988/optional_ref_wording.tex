\documentclass[a4paper,10pt,oneside,openany,final,article]{memoir}
%% Common header for WG21 proposals ? mainly taken from C++ standard draft source
%%

%%--------------------------------------------------
%% basics
% \documentclass[a4paper,11pt,oneside,openany,final,article]{memoir}

\usepackage[american]
           {babel}        % needed for iso dates
\usepackage[iso,american]
           {isodate}      % use iso format for dates
\usepackage[final]
           {listings}     % code listings
\usepackage{longtable}    % auto-breaking tables
\usepackage{ltcaption}    % fix captions for long tables
\usepackage{relsize}      % provide relative font size changes
\usepackage{textcomp}     % provide \text{l,r}angle
\usepackage{underscore}   % remove special status of '_' in ordinary text
\usepackage{parskip}      % handle non-indented paragraphs "properly"
\usepackage{array}        % new column definitions for tables
\usepackage[normalem]{ulem}
\usepackage{enumitem}
\usepackage{color}        % define colors for strikeouts and underlines
\usepackage{amsmath}      % additional math symbols
\usepackage{mathrsfs}     % mathscr font
\usepackage[final]{microtype}
\usepackage[splitindex,original]{imakeidx}
\usepackage{multicol}
\usepackage{lmodern}
\usepackage{xcolor}
\usepackage[T1]{fontenc}
\usepackage{graphicx}
\usepackage{hyperref}
% \usepackage[pdftex,
%             bookmarks=true,
%             bookmarksnumbered=true,
%             pdfpagelabels=true,
%             pdfpagemode=UseOutlines,
%             pdfstartview=FitH,
%             linktocpage=true,
%             colorlinks=true,
%             plainpages=false,
%             allcolors={blue},
%             allbordercolors={white}
%             ]{hyperref}

% \usepackage[pdftex,
%   pdfauthor={Steve Downey},
%   pdftitle={dXXXpX},
%   pdfkeywords={},
%   pdfsubject={},
%   pdfcreator={Emacs 28.1.50 (Org mode 9.5.2)},
%   pdflang={English}]{hyperref}

\usepackage{memhfixc}     % fix interactions between hyperref and memoir
\usepackage{expl3}
\usepackage{xparse}
\usepackage{xstring}

\usepackage{url}  % urls in ref.bib
\usepackage{tabularx}  % don't use the C++ standard's fancy tables, they come with captions!

\special{pdf:minorversion 5} %set minorversion
\special{pdf:compresslevel 9} %set minorversion
\special{pdf:objcompresslevel 9} %set minorversion

% \pdfminorversion=5
% \pdfcompresslevel=9
% \pdfobjcompresslevel=2

\renewcommand\RSsmallest{5.5pt}  % smallest font size for relsize

\input{stdtex/layout}
\input{stdtex/styles}
\input{stdtex/macros}
\input{stdtex/tables}

\usepackage{amssymb,graphicx,stackengine,xcolor}
\protected\def\ucr{\scalebox{1}{\stackinset{c}{}{c}{-.2pt}{%
      \textcolor{white}{\sffamily\bfseries\small ?}}{%
      \rotatebox{45}{$\blacksquare$}}}}

\newenvironment{wording}{
  \chapterstyle{cppstd}
  \newcounter{scratchChapter}\setcounter{scratchChapter}{\value{chapter}}
  \settocdepth{part}
  \renewcommand\thesection{\ucr.\arabic{section}}
  \renewcommand\thesubsection{\makebox{$\ucr$}.\makebox{$\ucr$}.\arabic{subsection}}
} {
  \setcounter{chapter}{\value{scratchChapter}}
  \settocdepth{section}

}


% %% --------------------------------------------------
% %% fix interaction between hyperref and other
% %% commands
% \pdfstringdefDisableCommands{\def\smaller#1{#1}}
% \pdfstringdefDisableCommands{\def\textbf#1{#1}}
% \pdfstringdefDisableCommands{\def\raisebox#1{}}
% \pdfstringdefDisableCommands{\def\hspace#1{}}



%%--------------------------------------------------
%% add special hyphenation rules
\hyphenation{tem-plate ex-am-ple in-put-it-er-a-tor name-space name-spaces non-zero}

%%--------------------------------------------------
%% turn off all ligatures inside \texttt
\DisableLigatures{encoding = T1, family = tt*}

%% --------------------------------------------------
%% configuration
\input{stdtex/config}

\settocdepth{chapter}
\usepackage{minted}
\usepackage{fontspec}
\setromanfont{Source Serif Pro}
\setsansfont{Source Sans Pro}
% \setmonofont{Source Code Pro}

\begin{document}
\title{std::optional<T\&>}
\author{
  Steve Downey \small<\href{mailto:sdowney@gmail.com}{sdowney@gmail.com}> \\
  Peter Sommerlad \small<\href{mailto:peter.cpp@sommerlad.ch}{peter.cpp@sommerlad.ch}> \\
}
\date{} %unused. Type date explicitly below.
\maketitle

\begin{flushright}
  \begin{tabular}{ll}
    Document \#: & D2988R7 \\
    Date: & \today \\
    Project: & Programming Language C++ \\
    Audience: & LEWG
  \end{tabular}
\end{flushright}

\begin{abstract}
  We propose to fix a hole intentionally left in \tcode{std::optional} ---

  An optional over a reference such that the post condition on assignment is independent of the engaged state, always producing a rebound reference, and assigning a \tcode{U} to a \tcode{T} is disallowed by \tcode{static_assert} if a \tcode{U} can not be bound to a \tcode{T\&}.
\end{abstract}

\tableofcontents*

\chapter*{Changes Since Last Version}

\begin{itemize}
  \item \textbf{Changes since R6}
  \begin{itemize}
  \item strike refref specialization
  \item add converting assignment operator
  \item add converting in place constructor
  \end{itemize}
  \item \textbf{Changes since R5}
  \begin{itemize}
  \item refref specialization
  \item fix monadic constraints on base template
  \end{itemize}
\item \textbf{Changes since R4}
  \begin{itemize}
  \item feature test macro
  \item value_or updates from P3091
  \end{itemize}
\item \textbf{Changes since R3}
  \begin{itemize}
  \item make_optional discussion - always value
  \item value_or discussion - always value
  \end{itemize}
\end{itemize}

\chapter{Comparison table}
\section{Using a raw pointer result for an element search function}

This is the convention the C++ core guidelines suggest, to use a raw pointer for representing optional non-owning references.
However, there is a user-required check against `nullptr`, no type safety meaning no safety against mis-interpreting such a raw pointer, for example by using pointer arithmetic on it.

\begin{tabular}{ lr }
  \begin{minipage}[t]{0.45\columnwidth}
    \begin{minted}[fontsize=\small]{c++}
      Cat* cat = find_cat("Fido");
      if (cat!=nullptr) { return doit(*cat); }


    \end{minted}
  \end{minipage}
  &
    \begin{minipage}[t]{0.45\columnwidth}
      \begin{minted}[fontsize=\small]{c++}
        std::optional<Cat&> cat = find_cat("Fido");
        return cat.and_then(doit);

      \end{minted}
    \end{minipage}
\end{tabular}

\section{returning result of an element search function via a (smart) pointer}

The disadvantage here is that \tcode{std::experimental::observer_ptr<T>} is both non-standard and not well named, therefore this example uses \tcode{shared_ptr} that would have the advantage of avoiding dangling through potential lifetime extension.
However, on the downside is still the explicit checks against the \tcode{nullptr} on the client side, failing so risks undefined behavior.

  \begin{tabular}{ lr }
  \begin{minipage}[t]{0.45\columnwidth}
    \begin{minted}[fontsize =\small]{c++}
std::shared_ptr<Cat> cat = find_cat("Fido");
if (cat != nullptr) {/* ... */}

    \end{minted}
  \end{minipage}
  &
    \begin{minipage}[t]{0.45\columnwidth}
      \begin{minted}[fontsize=\small]{c++}
std::optional<Cat&> cat = find_cat("Fido");
cat.and_then([](Cat& thecat){/* ... */}

      \end{minted}
    \end{minipage}
  \end{tabular}
  \section{returning result of an element search function via an iterator}

  This might be the obvious choice, for example, for associative containers, especially since their iterator stability guarantees.
  However, returning such an iterator will leak the underlying container type as well necessarily requires one to know the sentinel of the container to check for the not-found case.

  \begin{tabular}{ lr }
  \begin{minipage}[t]{0.45\columnwidth}
    \begin{minted}[fontsize=\small]{c++}
std::map<std::string, Cat>::iterator cat
        = find_cat("Fido");
if (cat != theunderlyingmap.end()){/* ... */}

    \end{minted}
  \end{minipage}
  &
    \begin{minipage}[t]{0.45\columnwidth}
      \begin{minted}[fontsize=\small]{c++}
std::optional<Cat&> cat
        = find_cat("Fido");
cat.and_then([](Cat& thecat){/* ... */}

      \end{minted}
    \end{minipage}
  \end{tabular}

  \section{Using an optional<T*> as a substitute for optional<T\&>}

This approach adds another level of indirection and requires two checks to take a definite action.

  \begin{tabular}{ lr }
  \begin{minipage}[t]{0.45\columnwidth}
    \begin{minted}[fontsize=\small]{c++}
//Mutable optional
std::optional<Cat*> c = find_cat("Fido");
if (c) {
    if (*c) {
      *c.value() = Cat("Fynn", color::orange);
    }
}

    \end{minted}
  \end{minipage}
  &
    \begin{minipage}[t]{0.45\columnwidth}
      \begin{minted}[fontsize=\small]{c++}
std::optional<Cat&> c = find_cat("Fido");
if (c) {
  *c = Cat("Fynn", color::orange);
}

//or

o.transform([](Cat& c){
  c = Cat("Fynn", color::orange);
});
        \end{minted}
      \end{minipage}
\end{tabular}

\chapter{Motivation}
Other than the standard library's implementation of optional, optionals holding references are common. The desire for such a feature is well understood, and many optional types in commonly used libraries provide it, with the semantics proposed here.
One standard library implementation already provides an implementation of \tcode{std::optional<T\&>} but disables its use, because the standard forbids it.

The research in JeanHeyd Meneide's _References for Standard Library Vocabulary Types - an optional case study._ \cite{P1683R0} shows conclusively that rebind semantics are the only safe semantic as assign through on engaged is too bug-prone. Implementations that attempt assign-through are abandoned. The standard library should follow existing practice and supply an \tcode{optional<T\&>} that rebinds on assignment.

Additional background reading on \tcode{optional<T\&>} can be found in JeanHeyd Meneide's article _To Bind and Loose a Reference_ \cite{REFBIND}.

In freestanding environments or for safety-critical libraries, an optional type over references is important to implement containers, that otherwise as the standard library either would cause undefined behavior when accessing an non-available element, throw an exception, or silently create the element. Returning a plain pointer for such an optional reference, as the core guidelines suggest, is a non-type-safe solution and doesn't protect in any way from accessing an non-existing element by a \tcode{nullptr} de-reference. In addition, the monadic APIs of \tcode{std::optional} makes is especially attractive by streamlining client code receiving such an optional reference, in contrast to a pointer that requires an explicit nullptr check and de-reference.

There is a principled reason not to provide a partial specialization over \tcode{T\&} as the semantics are in some ways subtly different than the primary template. Assignment may have side-effects not present in the primary, which has pure value semantics. However, I argue this is misleading, as reference semantics often has side-effects. The proposed semantic is similar to what an \tcode{optional<std::reference_wrapper<T>>} provides, with much greater usability.

There are well motivated suggestions that perhaps instead of an \tcode{optional<T\&>} there should be an \tcode{optional_ref<T>} that is an independent primary template. This proposal rejects that, because we need a policy over all sum types as to how reference semantics should work, as optional is a variant over T and monostate. That the library sum type can not express the same range of types as the product type, tuple, is an increasing problem as we add more types logically equivalent to a variant. The template types \tcode{optional} and \tcode{expected} should behave as extensions of \tcode{variant<T, monostate>} and \tcode{variant<T, E>}, or we lose the ability to reason about generic types.

That we can't guarantee from \tcode{std::tuple<Args...>} (product type) that \tcode{std::variant<Args...>} (sum type) is valid, is a problem, and one that reflection can't solve. A language sum type could, but we need agreement on the semantics.

The semantics of a variant with a reference are as if it holds the address of the referent when referring to that referent. All other semantics are worse. Not being able to express a variant<T\&> is inconsistent, hostile, and strictly worse than disallowing it.

Thus, we expect future papers to propose \tcode{std::expected<T\&,E>} and \tcode{std::variant} with the ability to hold references.
The latter can be used as an iteration type over \tcode{std::tuple} elements.


\chapter{Design}

The design is straightforward. The \tcode{optional<T\&>} holds a pointer to the underlying object of type \tcode{T}, or \tcode{nullptr} if the optional is disengaged. The implementation is simple, especially with C++20 and up techniques, using concept constraints. As the held pointer is a primitive regular type with reference semantics, many operations can be defaulted and are \tcode{noexcept} by nature. See \cite{Downey_smd_optional_optional_T} and \cite{rawgithu58:online}. The \tcode{optional<T\&>} implementation is less than 200 lines of code, much of it the monadic functions with identical textual implementations with different signatures and different overloads being called.

In place construction is not supported as it would just be a way of providing immediate life-time issues.

\section{Relational Operations}

The definitions of the relational operators are the same as for the base template. Interoperable comparisons between T and optional<T\&> work as expected. This is not true for the boost optional<T\&>.

\section{make_optional}
Because of existing code, \tcode{make_optional<T\&>} must return optional<T> rather than optional<T\&>. Returning optional<T\&> is consistent and defensible, and a few optional implementations in production make this choice. It is, however, quite easy to construct a make_optional expression that deduces a different category causing possibly dangerous changes to code.

There was some discussion about using library technology to allow selection of the reference overload via the literal spelling \tcode{make_optional<int\&>}. There was anti-consensus to do so. There are existing instances of that spelling that today return an \tcode{optional<T>}, although it is very likely these are mistakes or possibly other optionals. The spelling \tcode{optional<T\&>\{\}} is acceptable as there is no multi-argument emplacement version as there is no location to construct such an instance.

There was also discussion of using \tcode{std::reference_wrapper} to indicate reference use, in analogy with std::tuple. Unfortunately there are existing uses of optional over reference_wrapper as a workaround for lack of reference specialization, and it would be a breaking change for such code.

\section{Trivial construction}
Construction of \tcode{optional<T\&>} should be trivial, because it is straightforward to implement, and \tcode{optional<T>} is trivial. Boost is not.

\section{Value Category Affects value()}
For several implementations there are distinct overloads for functions depending on value category, with the same implementation. However, this makes it very easy to accidentally steal from the underlying referred to object. Value category should be shallow. Thanks to many people for pointing this out. If ``Deducing \tcode{this}'' had been used, the problem would have been much more subtle in code review.

\section{Shallow vs Deep const}

There is some implementation divergence in optionals about deep const for \tcode{optional<T\&>}. That is, can the referred to \tcode{int} be modified through a \tcode{const optional<int\&>}. Does \tcode{operator->()} return an \tcode{int*} or a \tcode{const int*}, and does \tcode{operator*()} return an \tcode{int\&} or a \tcode{const int\&}. I believe it is overall more defensible if the \tcode{const} is shallow as it would be for a \tcode{struct ref {int * p;}} where the constness of the struct ref does not affect if the p pointer can be written through. This is consistent with the rebinding behavior being proposed.

Where deeper constness is desired, \tcode{optional<const T\&>} would prevent non const access to the underlying object.

\section{Conditional Explicit}
As in the base template, \tcode{explicit} is made conditional on the type used to construct the optional. \tcode{explicit(!std::is_convertible_v<U, T>)}. This is not present in boost::optional, leading to differences in construction between braced initialization and = that can be surprising.

\section{value_or}
\removed{Have \tcode{value_or} return a \tcode{T\&}. Check that the supplied value can be bound to a T\&.}

After extensive discussion, it seems there is no particularly wonderful solution for \tcode{value_or} that does not involve a time machine. Implementations of optionals that support reference semantics diverge over the return type, and the current one is arguably wrong, and should use something based on \tcode{common_reference_type}, which of course did not exist when \tcode{optional} was standardized.

The weak consensus is to return a \tcode{T} from \tcode{optional<T\&>::value_or} as this is least likely to cause issues. There was at least one strong objection to this choice, but all other choices had more objections. The author intends to propose free functions \tcode{reference_or}, \tcode{value_or}, \tcode{or_invoke}, and \tcode{yield_if} over all types modeling optional-like, \tcode{concept std::maybe}, in the next revision of \cite{P1255R12}. This would cover \tcode{optional}, \tcode{expected}, and pointer types.

Having \tcode{value_or} return by value also allows the common case of using a literal as the alternative to be expressed concisely.

\section{in_place_t construction}
The reference specialization allows a limited form of in_place construction where the argument can be converted to the appropriate reference without creation of a temporary. As the reference specialization is non-owning, there is no ``place'' for a temporary to be constructed that will not dangle. For cases where the lifetime of the constructed object would match the lifetime of the optional, the temporary can be constructed explicitly, instead.

\section{Converting assignment}
A similarly limited converting assignment operator is provided for cases where an optional<U> has a value or refers to a value which can be converted to a T\& without construction of a temporary. In particular, converting an optional<T\&> to an optional<T const\&> is supported.

\section{Compiler Explorer Playground}

See \url{https://godbolt.org/z/n5oooK58W} for a playground with relevant Google Test functions and various optional implementations made available for cross reference.

\chapter{Proposal}

Add lvalue and rvalue reference specializations for the std::optional template.

\chapter{Wording}

The wording here cross references and adopts the wording in \cite{P3091R2}.


\begin{wording}

\rSec1[optional]{Optional objects}

\rSec2[optional.general]{General}

\pnum
Subclause~\ref{optional} describes class template \tcode{optional} that represents
optional objects.
An \defn{optional object} is an
object that contains the storage for another object and manages the lifetime of
this contained object, if any. The contained object may be initialized after
the optional object has been initialized, and may be destroyed before the
optional object has been destroyed. The initialization state of the contained
object is tracked by the optional object.

\rSec2[optional.syn]{Header \tcode{<optional>} synopsis}

\indexheader{optional}%
\begin{codeblock}
// mostly freestanding
#include <compare>              // see \ref{compare.syn}

namespace std {
  // \ref{optional.optional}, class template \tcode{optional} for object types
  template <class T>
  class optional; // partially freestanding

  @\added{// \ref{optional.optionalref}, class template \tcode{optional} for reference types }@
  @\added{template <class T> }@
  @\added{class optional<T\&>; // partially freestanding}@

  template <class T>
  inline constexpr bool ranges::enable_view<optional<T>> = true;
  template <class T>
  inline constexpr auto format_kind<optional<T>> = range_format::disabled;

  template<class T>
    concept @\defexposconcept{is-derived-from-optional}@ = requires(const T& t) {       // \expos
      []<class U>(const optional<U>&){ }(t);
    };

  // \ref{optional.nullopt}, no-value state indicator
  struct nullopt_t{@\seebelow@};
  inline constexpr nullopt_t nullopt(@\unspec@);

  // \ref{optional.bad.access}, class \tcode{bad_optional_access}
  class bad_optional_access;

  // \ref{optional.relops}, relational operators
  template<class T, class U>
    constexpr bool operator==(const optional<T>&, const optional<U>&);
  template<class T, class U>
    constexpr bool operator!=(const optional<T>&, const optional<U>&);
  template<class T, class U>
    constexpr bool operator<(const optional<T>&, const optional<U>&);
  template<class T, class U>
    constexpr bool operator>(const optional<T>&, const optional<U>&);
  template<class T, class U>
    constexpr bool operator<=(const optional<T>&, const optional<U>&);
  template<class T, class U>
    constexpr bool operator>=(const optional<T>&, const optional<U>&);
  template<class T, @\libconcept{three_way_comparable_with}@<T> U>
    constexpr compare_three_way_result_t<T, U>
      operator<=>(const optional<T>&, const optional<U>&);

  // \ref{optional.nullops}, comparison with \tcode{nullopt}
  template<class T> constexpr bool operator==(const optional<T>&, nullopt_t) noexcept;
  template<class T>
    constexpr strong_ordering operator<=>(const optional<T>&, nullopt_t) noexcept;

  // \ref{optional.comp.with.t}, comparison with \tcode{T}
  template<class T, class U> constexpr bool operator==(const optional<T>&, const U&);
  template<class T, class U> constexpr bool operator==(const T&, const optional<U>&);
  template<class T, class U> constexpr bool operator!=(const optional<T>&, const U&);
  template<class T, class U> constexpr bool operator!=(const T&, const optional<U>&);
  template<class T, class U> constexpr bool operator<(const optional<T>&, const U&);
  template<class T, class U> constexpr bool operator<(const T&, const optional<U>&);
  template<class T, class U> constexpr bool operator>(const optional<T>&, const U&);
  template<class T, class U> constexpr bool operator>(const T&, const optional<U>&);
  template<class T, class U> constexpr bool operator<=(const optional<T>&, const U&);
  template<class T, class U> constexpr bool operator<=(const T&, const optional<U>&);
  template<class T, class U> constexpr bool operator>=(const optional<T>&, const U&);
  template<class T, class U> constexpr bool operator>=(const T&, const optional<U>&);
  template<class T, class U>
      requires (!@\exposconcept{is-derived-from-optional}@<U>) && @\libconcept{three_way_comparable_with}@<T, U>
    constexpr compare_three_way_result_t<T, U>
      operator<=>(const optional<T>&, const U&);

  // \ref{optional.specalg}, specialized algorithms
  template<class T>
    constexpr void swap(optional<T>&, optional<T>&) noexcept(@\seebelow@);

  template<class T>
    constexpr optional<@\seebelow@> make_optional(T&&);
  template<class T, class... Args>
    constexpr optional<T> make_optional(Args&&... args);
  template<class T, class U, class... Args>
    constexpr optional<T> make_optional(initializer_list<U> il, Args&&... args);

  // \ref{optional.hash}, hash support
  template<class T> struct hash;
  template<class T> struct hash<optional<T>>;
}
\end{codeblock}

\rSec2[optional.optional]{Class template \tcode{optional} \added{for object types}}

\rSec3[optional.optional.general]{General}

\indexlibraryglobal{optional}%
\indexlibrarymember{value_type}{optional}%
\begin{codeblock}
namespace std {
  template<class T>
  class optional {
  public:
    using value_type     = T;
    using iterator       = @\impdefnc@;              // see~\ref{optional.iterators}
    using const_iterator = @\impdefnc@;              // see~\ref{optional.iterators}

    // \ref{optional.ctor}, constructors
    constexpr optional() noexcept;
    constexpr optional(nullopt_t) noexcept;
    constexpr optional(const optional&);
    constexpr optional(optional&&) noexcept(@\seebelow@);

    template<class... Args>
      constexpr explicit optional(in_place_t, Args&&...);
    template<class U, class... Args>
      constexpr explicit optional(in_place_t, initializer_list<U>, Args&&...);
    template<class U = T>
      constexpr explicit(@\seebelow@) optional(U&&);
    template<class U>
      constexpr explicit(@\seebelow@) optional(const optional<U>&);
    template<class U>
      constexpr explicit(@\seebelow@) optional(optional<U>&&);

    // \ref{optional.dtor}, destructor
    constexpr ~optional();

    // \ref{optional.assign}, assignment
    constexpr optional& operator=(nullopt_t) noexcept;
    constexpr optional& operator=(const optional&);
    constexpr optional& operator=(optional&&) noexcept(@\seebelow@);
    template<class U = T> constexpr optional& operator=(U&&);
    template<class U> constexpr optional& operator=(const optional<U>&);
    template<class U> constexpr optional& operator=(optional<U>&&);
    template<class... Args> constexpr T& emplace(Args&&...);
    template<class U, class... Args> constexpr T& emplace(initializer_list<U>, Args&&...);

    // \ref{optional.swap}, swap
    constexpr void swap(optional&) noexcept(@\seebelow@);

    // \ref{optional.iterators}, iterator support
    constexpr iterator begin() noexcept;
    constexpr const_iterator begin() const noexcept;
    constexpr iterator end() noexcept;
    constexpr const_iterator end() const noexcept;

    // \ref{optional.observe}, observers
    constexpr const T* operator->() const noexcept;
    constexpr T* operator->() noexcept;
    constexpr const T& operator*() const & noexcept;
    constexpr T& operator*() & noexcept;
    constexpr T&& operator*() && noexcept;
    constexpr const T&& operator*() const && noexcept;
    constexpr explicit operator bool() const noexcept;
    constexpr bool has_value() const noexcept;
    constexpr const T& value() const &;                                 // freestanding-deleted
    constexpr T& value() &;                                             // freestanding-deleted
    constexpr T&& value() &&;                                           // freestanding-deleted
    constexpr const T&& value() const &&;                               // freestanding-deleted
    template<class U> constexpr T value_or(U&&) const &;
    template<class U> constexpr T value_or(U&&) &&;

    // \ref{optional.monadic}, monadic operations
    template<class F> constexpr auto and_then(F&& f) &;
    template<class F> constexpr auto and_then(F&& f) &&;
    template<class F> constexpr auto and_then(F&& f) const &;
    template<class F> constexpr auto and_then(F&& f) const &&;
    template<class F> constexpr auto transform(F&& f) &;
    template<class F> constexpr auto transform(F&& f) &&;
    template<class F> constexpr auto transform(F&& f) const &;
    template<class F> constexpr auto transform(F&& f) const &&;
    template<class F> constexpr optional or_else(F&& f) &&;
    template<class F> constexpr optional or_else(F&& f) const &;

    // \ref{optional.mod}, modifiers
    constexpr void reset() noexcept;

  private:
    T *val;         // \expos
  };

  template<class T>
    optional(T) -> optional<T>;
}
\end{codeblock}

\pnum
Any instance of \tcode{optional<T>} at any given time either contains a value or does not contain a value.
When an instance of \tcode{optional<T>} \defnx{contains a value}{contains a value!\idxcode{optional}},
it means that an object of type \tcode{T}, referred to as the optional object's \defnx{contained value}{contained value!\idxcode{optional}},
is allocated within the storage of the optional object.
Implementations are not permitted to use additional storage, such as dynamic memory, to allocate its contained value.
When an object of type \tcode{optional<T>} is contextually converted to \tcode{bool},
the conversion returns \tcode{true} if the object contains a value;
otherwise the conversion returns \tcode{false}.

\pnum
When an \tcode{optional<T>} object contains a value,
member \tcode{val} points to the contained value.

\pnum
\tcode{T} shall be a \added{non-array object} type
other than \cv{} \tcode{in_place_t} or \cv{} \tcode{nullopt_t}
that meets the \oldconcept{Destructible} requirements (\tref{cpp17.destructible}).

\rSec3[optional.ctor]{Constructors}


\rSec3[optional.dtor]{Destructor}

\rSec3[optional.assign]{Assignment}


\rSec3[optional.swap]{Swap}


\rSec3[optional.iterators]{Iterator support}

\rSec3[optional.observe]{Observers}

\rSec3[optional.monadic]{Monadic operations}

\indexlibrarymember{and_then}{optional}
\begin{itemdecl}
template<class F> constexpr auto and_then(F&& f) &;
template<class F> constexpr auto and_then(F&& f) const &;
\end{itemdecl}

\begin{itemdescr}
\pnum
Let \tcode{U} be \tcode{invoke_result_t<F, decltype(*val)>}.

\pnum
\mandates
\tcode{remove_cvref_t<U>} is a specialization of \tcode{optional}.

\pnum
\effects
Equivalent to:
\begin{codeblock}
if (*this) {
  return invoke(std::forward<F>(f), *val);
} else {
  return remove_cvref_t<U>();
}
\end{codeblock}
\end{itemdescr}

\indexlibrarymember{and_then}{optional}
\begin{itemdecl}
template<class F> constexpr auto and_then(F&& f) &&;
template<class F> constexpr auto and_then(F&& f) const &&;
\end{itemdecl}

\begin{itemdescr}
\pnum
Let \tcode{U} be \tcode{invoke_result_t<F, decltype(std::move(*val))>}.

\pnum
\mandates
\tcode{remove_cvref_t<U>} is a specialization of \tcode{optional}.

\pnum
\effects
Equivalent to:
\begin{codeblock}
if (*this) {
  return invoke(std::forward<F>(f), std::move(*val));
} else {
  return remove_cvref_t<U>();
}
\end{codeblock}
\end{itemdescr}

\indexlibrarymember{transform}{optional}
\begin{itemdecl}
template<class F> constexpr auto transform(F&& f) &;
template<class F> constexpr auto transform(F&& f) const &;
\end{itemdecl}

\begin{itemdescr}
\pnum
Let \tcode{U} be \tcode{remove_cv_t<invoke_result_t<F, decltype(*val)>>}.

\pnum
\mandates
\tcode{U} is a \added{reference or} non-array object type
other than \tcode{in_place_t} or \tcode{nullopt_t}.
The declaration
\begin{codeblock}
U u(invoke(std::forward<F>(f), *val));
\end{codeblock}
is well-formed for some invented variable \tcode{u}.
\begin{note}
There is no requirement that \tcode{U} is movable\iref{dcl.init.general}.
\end{note}

\pnum
\returns
If \tcode{*this} contains a value, an \tcode{optional<U>} object
whose contained value is direct-non-list-initialized with
\tcode{invoke(std::forward<F>(f), *val)};
otherwise, \tcode{optional<U>()}.
\end{itemdescr}

\indexlibrarymember{transform}{optional}
\begin{itemdecl}
template<class F> constexpr auto transform(F&& f) &&;
template<class F> constexpr auto transform(F&& f) const &&;
\end{itemdecl}

\begin{itemdescr}
\pnum
Let \tcode{U} be
\tcode{remove_cv_t<invoke_result_t<F, decltype(std::move(*val))>>}.

\pnum
\mandates
\tcode{U} is a \added{reference or} non-array object type
other than \tcode{in_place_t} or \tcode{nullopt_t}.
The declaration
\begin{codeblock}
U u(invoke(std::forward<F>(f), std::move(*val)));
\end{codeblock}
is well-formed for some invented variable \tcode{u}.
\begin{note}
There is no requirement that \tcode{U} is movable\iref{dcl.init.general}.
\end{note}

\pnum
\returns
If \tcode{*this} contains a value, an \tcode{optional<U>} object
whose contained value is direct-non-list-initialized with
\tcode{invoke(std::forward<F>(f), std::move(*val))};
otherwise, \tcode{optional<U>()}.
\end{itemdescr}

\indexlibrarymember{or_else}{optional}
\begin{itemdecl}
template<class F> constexpr optional or_else(F&& f) const &;
\end{itemdecl}

\begin{itemdescr}
\pnum
\constraints
\tcode{F} models \tcode{\libconcept{invocable}<>} and
\tcode{T} models \libconcept{copy_constructible}.

\pnum
\mandates
\tcode{is_same_v<remove_cvref_t<invoke_result_t<F>>, optional>} is \tcode{true}.

\pnum
\effects
Equivalent to:
\begin{codeblock}
if (*this) {
  return *this;
} else {
  return std::forward<F>(f)();
}
\end{codeblock}
\end{itemdescr}

\indexlibrarymember{or_else}{optional}
\begin{itemdecl}
template<class F> constexpr optional or_else(F&& f) &&;
\end{itemdecl}

\begin{itemdescr}
\pnum
\constraints
\tcode{F} models \tcode{\libconcept{invocable}<>} and
\tcode{T} models \libconcept{move_constructible}.

\pnum
\mandates
\tcode{is_same_v<remove_cvref_t<invoke_result_t<F>>, optional>} is \tcode{true}.

\pnum
\effects
Equivalent to:
\begin{codeblock}
if (*this) {
  return std::move(*this);
} else {
  return std::forward<F>(f)();
}
\end{codeblock}
\end{itemdescr}

\rSec3[optional.mod]{Modifiers}

\begin{addedblock}
\rSec2[optional.optionalref]{Class template \tcode{optional} for reference types}

\rSec3[optional.optionalref.general]{General}
\begin{codeblock}
namespace std {
  template<class T>
  class optional<T&> {
    public:
      using value_type     = T;
      using iterator       = @\impdefnc@; // see~\ref{optionalref.iterators}

    public:
      // \ref{optionalref.ctor}, constructors
      constexpr optional() noexcept;
      constexpr optional(nullopt_t) noexcept;
      constexpr optional(const optional& rhs) noexcept = default;

      template <class Arg>
      constexpr explicit optional(in_place_t, Arg&& arg);

      template <class U> constexpr explicit(@\seebelow@) optional(U&& u) noexcept(@\seebelow@);

      template <class U> constexpr explicit(@\seebelow@) optional(optional<U>& rhs) noexcept(@\seebelow@)
      template <class U> constexpr explicit(@\seebelow@) optional(const optional<U>& rhs) noexcept(@\seebelow@)
      template <class U> constexpr explicit(@\seebelow@) optional(optional<U>&& rhs) noexcept(@\seebelow@)
      template <class U> constexpr explicit(@\seebelow@) optional(const optional<U>&&& rhs) noexcept(@\seebelow@)

      constexpr ~optional() = default;

      // \ref{optionalref.assign}, assignment
      constexpr optional& operator=(nullopt_t) noexcept;
      constexpr optional& operator=(const optional& rhs) noexcept = default;

      template <class U> constexpr optional& emplace(U&& u) noexcept(@\seebelow@);

      // \ref{optionalref.swap}, swap
      constexpr void swap(optional& rhs) noexcept;

      // \ref{optional.iterators}, iterator support
      constexpr iterator begin() const noexcept;
      constexpr iterator end() const noexcept;

      // \ref{optionalref.observe}, observers
      constexpr T*       operator->() const noexcept;
      constexpr T&       operator*() const noexcept;
      constexpr explicit operator bool() const noexcept;
      constexpr bool     has_value() const noexcept;
      constexpr T&       value() const;                                // freestanding-deleted
      template <class U> constexpr remove_cv_t<T> value_or(U&& u) const;

      // \ref{optionalref.monadic}, monadic operations
      template <class F> constexpr auto and_then(F&& f) const;
      template <class F> constexpr auto transform(F&& f) const -> optional<invoke_result_t<F, T&>>;
      template <class F> constexpr optional or_else(F&& f) const;

      // \ref{optionalref.mod}, modifiers
      constexpr void reset() noexcept;

    private:
      T* val; // \expos
  };

}
\end{codeblock}
\rSec3[optionalref.ctor]{Constructors}

\begin{itemdecl}
constexpr optional() noexcept;
constexpr optional(nullopt_t) noexcept;
\end{itemdecl}

\begin{itemdescr}
\pnum
\ensures
\tcode{*this} does not refer to a value.

\pnum
\remarks
    For all \tcode{T\&} these constructors are constexpr constructors\iref{dcl.constexpr}.
\end{itemdescr}


\begin{itemdecl}
template <class Arg>
constexpr explicit optional(in_place_t, Arg&& arg);
\end{itemdecl}

\begin{itemdescr}
  \pnum
  \constraints
  \begin{itemize}
  \item \tcode{is_constructible_v<add_lvalue_reference_t<T>, Arg>} is \tcode{true}
  \end{itemize}
  \pnum
  \mandates
    \begin{itemize}
    \item \tcode{reference_converts_from_temporary_v<R, Arg>} is \tcode{false}
    \end{itemize}
  \pnum
  \ensures
  \tcode{*this} refers to the T created by conversion from \tcode{arg}

\end{itemdescr}

\begin{itemdecl}
template <class U>
  constexpr explicit(@\seebelow@) optional(U&& u) noexcept;
\end{itemdecl}

\begin{itemdescr}
  \pnum
  \constraints
  \begin{itemize}
  \item \tcode{!\exposconcept{is-derived-from-optional}<decay_t<U>>} is \tcode{true}
  \end{itemize}
    \pnum
    \mandates
    \begin{itemize}
    \item \tcode{is_constructible_v<add_lvalue_reference_t<T>, U>} is \tcode{true}
    \item \tcode{is_lvalue_reference<U>::value>} is \tcode{true}
    \end{itemize}

    \pnum
    \effects
    Direct-non-list-initializes \tcode{val} with \tcode{addressof(u)}.

    \pnum
    \ensures
    \tcode{*this} refers to a value.\end{itemdescr}

  \pnum
  \remarks
  The expression inside \keyword{explicit} is equivalent to:
  \begin{codeblock}
    !is_convertible_v<U, T>
  \end{codeblock}

\begin{itemdecl}
template <class U>
constexpr explicit(@\seebelow@) optional(const optional<U>& rhs) noexcept;
\end{itemdecl}

\begin{itemdescr}
  \pnum
  \mandates
    \begin{itemize}
    \item \tcode{is_constructible_v<add_lvalue_reference_t<T>, U>} is \tcode{true}
    \end{itemize}
  \pnum
  \ensures
  If \tcode{rhs.has_value()} is \tcode{true}, \tcode{*this} refers to the same value, otherwise, \tcode{*this} does not refer to a value.
  \pnum
  \remarks
  The expression inside \keyword{explicit} is equivalent to:
  \begin{codeblock}
    !is_convertible_v<U, T>
  \end{codeblock}

\end{itemdescr}


\rSec3[optionalref.assign]{Assignment}

\begin{itemdecl}
constexpr optional& operator=(nullopt_t) noexcept;
\end{itemdecl}

\begin{itemdescr}
  \pnum
  \ensures
  \tcode{*this} does not refer to a value.

  \pnum
  \returns
  \tcode{*this}.
\end{itemdescr}

\begin{itemdecl}
template <class U = T>
constexpr optional& operator=(U&& u);
\end{itemdecl}

\begin{itemdescr}
  \pnum
  \constraints
  \begin{itemize}
  \item \tcode{!\exposconcept{is-derived-from-optional}<decay_t<U>>} is \tcode{true}
  \end{itemize}

  \pnum
  \mandates
  \begin{itemize}
  \item \tcode{is_constructible_v<add_lvalue_reference_t<T>, U>} is \tcode{true}
  \item \tcode{is_lvalue_reference<U>::value>} is \tcode{true}
  \end{itemize}

  \pnum
  \effects
  Assigns the \tcode{val} the value of \tcode{addressof(v)}.

  \pnum
  \ensures
  \tcode{*this} refers to a value.
\end{itemdescr}

\begin{itemdecl}
template <class U>
constexpr optional& operator=(const optional<U>& rhs) noexcept;
\end{itemdecl}

\begin{itemdescr}
    \pnum
    \mandates
    \begin{itemize}
    \item \tcode{is_constructible_v<add_lvalue_reference_t<T>, U>} is \tcode{true}
    \item \tcode{is_lvalue_reference<U>::value>} is \tcode{true}
    \end{itemize}

    \pnum
    \ensures
    \tcode{rhs.has_value() == this->has_value()}.

    \pnum
    \returns
    \tcode{*this}.
\end{itemdescr}

\begin{itemdecl}
template <class U>
constexpr optional& operator=(optional<U>&& rhs) noexcept;
\end{itemdecl}

\begin{itemdescr}
    \pnum
    \mandates
    \begin{itemize}
    \item \tcode{is_constructible_v<add_lvalue_reference_t<T>, U>} is \tcode{true}
    \item \tcode{reference_converts_from_temporary_v<R, Arg>} is \tcode{false}
    \end{itemize}

    \pnum
    \ensures
    If \tcode{rhs.has_value()} is \tcode{true}, \tcode{*this} refers to the value obtained by converting the \tcode{rhs} referred to value to a \tcode{T\&}, otherwise, \tcode{*this} does not refer to a value

    \pnum
    \returns
    \tcode{*this}.
\end{itemdescr}

\begin{itemdecl}
template <class U>
constexpr optional& emplace(U&& u) noexcept;
\end{itemdecl}

\begin{itemdescr}
  \pnum
  \constraints
  \begin{itemize}
  \item \tcode{!\exposconcept{is-derived-from-optional}<decay_t<U>>} is \tcode{true}
  \end{itemize}

  \pnum
  \effects
  Assigns \tcode{*this} \tcode{forward<U>(u)}

  \pnum
  \ensures
  \tcode{*this} refers to a value.
\end{itemdescr}


\rSec3[optionalref.swap]{Swap}

\begin{itemdecl}
constexpr void swap(optional& rhs) noexcept;
\end{itemdecl}

\begin{itemdescr}
  \pnum
  \effects
  *this and *rhs will refer to each others initial referred to objects.
\end{itemdescr}


\rSec3[optionalref.iterators]{Iterator support}
\begin{itemdecl}
using iterator = @\impdef@;
using const_iterator = @\impdef@;
\end{itemdecl}

\begin{itemdescr}
\pnum
These types
model \libconcept{contiguous_iterator}\iref{iterator.concept.contiguous},
meet the \oldconcept{RandomAccessIterator} requirements\iref{random.access.iterators}, and
meet the requirements for constexpr iterators\iref{iterator.requirements.general},
with value type \tcode{remove_cv_t<T>}.
The reference type is \tcode{T\&} for \tcode{iterator} and
\tcode{const T\&} for \tcode{const_iterator}.

\pnum
All requirements on container iterators\iref{container.reqmts} apply to
\tcode{optional::iterator} and \tcode{optional::\linebreak{}const_iterator} as well.

\end{itemdescr}


\begin{itemdecl}
constexpr iterator begin() noexcept;
constexpr const_iterator begin() const noexcept;
\end{itemdecl}

\begin{itemdescr}
  \pnum
  \returns
  If \tcode{has_value()} is \tcode{true},
  an iterator referring to the referred to value.
  Otherwise, a past-the-end iterator value.

\end{itemdescr}

\begin{itemdecl}
constexpr iterator end() noexcept;
constexpr const_iterator end() const noexcept;
\end{itemdecl}

\begin{itemdescr}
\pnum
\returns
\tcode{begin() + has_value()}.
\end{itemdescr}



\rSec3[optionalref.observe]{Observers}

\begin{itemdecl}
constexpr T* operator->() const noexcept;
\end{itemdecl}

\begin{itemdescr}
  \pnum
  \returns
  \tcode{val}.

\end{itemdescr}

\begin{itemdecl}
constexpr T& operator*() const noexcept;
\end{itemdecl}

\begin{itemdescr}
   \pnum
   \expects
   \tcode{*this} refers to a value.

   \pnum
   \returns
   \tcode{*val}.

\end{itemdescr}

\begin{itemdecl}
constexpr explicit operator bool() const noexcept;
\end{itemdecl}

\begin{itemdescr}
  \pnum
  \returns
  \tcode{true} if and only if \tcode{*this} refers to a value.
\end{itemdescr}

\begin{itemdecl}
constexpr bool has_value() const noexcept;
\end{itemdecl}

\begin{itemdescr}
  \pnum
  \returns
  \tcode{true} if and only if \tcode{*this} refers to a value.
\end{itemdescr}

\begin{itemdecl}
constexpr T& value() const;
\end{itemdecl}

\begin{itemdescr}
  \pnum
  \effects
  Equivalent to:
  \begin{codeblock}
    if (has_value())
      return *value_;
    throw bad_optional_access();
  \end{codeblock}
\end{itemdescr}

\begin{itemdecl}
template <class U>
constexpr T value_or(U&& u) const;
\end{itemdecl}

\begin{itemdescr}
  \pnum
  \mandates
  \tcode{is_constructible_v<add_lvalue_reference_t<T>, decltype(v)>} is \tcode{true}.

  \pnum
  \effects
  Equivalent to:
  \tcode{return has_value() ? U(**this) : U(std::forward<Args>(u));}
\end{itemdescr}


\rSec3[optionalref.monadic]{Monadic operations}

\begin{itemdecl}
template <class F>
constexpr auto and_then(F&& f) const;
\end{itemdecl}

\begin{itemdescr}
  \pnum
  Let \tcode{U} be \tcode{invoke_result_t<F, T\&>}.

  \pnum
  \mandates
  \tcode{remove_cvref_t<U>} is a specialization of \tcode{optional}.

  \pnum
  \effects
  Equivalent to:
  \begin{codeblock}
    return has_value() ? invoke(forward<F>(f), value()) : result(nullopt);
  \end{codeblock}
\end{itemdescr}

\begin{itemdecl}
template <class F>
constexpr auto transform(F&& f) const -> optional<invoke_result_t<F, T&>>;
\end{itemdecl}

\begin{itemdescr}
  \pnum
  Let \tcode{U} be \tcode{remove_cv_t<invoke_result_t<F, T\&>>}.


  \pnum
  \returns
  If \tcode{*this} refers to a value, an \tcode{optional<U>} object
  whose refered to value is the result of
  \tcode{invoke(forward<F>(f), *val)};
  otherwise, \tcode{optional<U>()}.
\end{itemdescr}

\begin{itemdecl}
template <class F>
constexpr optional or_else(F&& f) const;
\end{itemdecl}

\begin{itemdescr}
  \pnum
  \mandates
  \tcode{is_same_v<remove_cvref_t<invoke_result_t<F>>, optional>} is \tcode{true}.

  \pnum
  \effects
  Equivalent to:
  \begin{codeblock}
    if (has_value()))
        return value()
    else
        return forward<F>(f)();
  \end{codeblock}
\end{itemdescr}


\rSec3[optionalref.mod]{Modifiers}

\begin{itemdecl}
constexpr void reset() noexcept;
\end{itemdecl}

\begin{itemdescr}
  \pnum
  \ensures
  \tcode{*this} does not refer to a value.
\end{itemdescr}

\end{addedblock}

\rSec2[optional.nullopt]{No-value state indicator}
\rSec2[optional.bad.access]{Class \tcode{bad_optional_access}}
\rSec2[optional.relops]{Relational operators}
\rSec2[optional.specalg]{Specialized algorithms}
\rSec2[optional.hash]{Hash support}

\begin{addedblock}
\Sec2[version.syn]{Feature-test macro}
Add the following macro definition to [version.syn], header <version> synopsis, with the value selected by the editor to reflect the date of adoption of this paper:

\begin{codeblock}
  #define __cpp_lib_optional_ref 20XXXXL // also in <ranges>, <tuple>, <utility>
\end{codeblock}
\end{addedblock}


\end{wording}

\chapter{Impact on the standard}

A pure library extension, affecting no other parts of the library or language.

The proposed changes are relative to the current working draft \cite{N4910}.

\chapter*{Document history}

\begin{itemize}
\item \textbf{Changes since R1}
  \begin{itemize}
  \item Design points called out
  \end{itemize}
\item \textbf{Changes since R0}
  \begin{itemize}
  \item Wording Updates
  \end{itemize}
\end{itemize}

\renewcommand{\bibname}{References}
\bibliographystyle{abstract}
\bibliography{wg21,mybiblio}


\backmatter
\chapter*{Implementation}

\begin{minted}{c++}
  // ----------------------
  // BASE AND DETAILS ELIDED
  // ----------------------

  /****************/
  /* optional<T&> */
  /****************/

  template <class T>
  class optional<T&> {
    public:
    using value_type = T&;
    using iterator =
    detail::contiguous_iterator<T,
    optional>; // see [optionalref.iterators]
    using const_iterator =
    detail::contiguous_iterator<const T,
    optional>; // see [optionalref.iterators]

    private:
    template <class R, class Arg>
    constexpr R make_reference(Arg&& arg) // exposition only
    requires is_constructible_v<R, Arg>;

    public:
    // \ref{optionalref.ctor}, constructors

    constexpr optional() noexcept;
    constexpr optional(nullopt_t) noexcept;
    constexpr optional(const optional& rhs) noexcept = default;
    constexpr optional(optional&& rhs) noexcept      = default;

    template <class Arg>
    constexpr explicit optional(in_place_t, Arg&& arg)
    requires is_constructible_v<add_lvalue_reference_t<T>, Arg>;

    template <class U = T>
    requires(!detail::is_optional<decay_t<U>>)
    constexpr explicit(!is_convertible_v<U, T>) optional(U&& u) noexcept;
    template <class U>
    constexpr explicit(!is_convertible_v<U, T>)
    optional(const optional<U>& rhs) noexcept;

    // \ref{optionalref.dtor}, destructor
    constexpr ~optional() = default;

    // \ref{optionalref.assign}, assignment
    constexpr optional& operator=(nullopt_t) noexcept;

    constexpr optional& operator=(const optional& rhs) noexcept = default;
    constexpr optional& operator=(optional&& rhs) noexcept      = default;

    template <class U = T>
    requires(!detail::is_optional<decay_t<U>>)
    constexpr optional& operator=(U&& u);

    template <class U>
    constexpr optional& operator=(const optional<U>& rhs) noexcept;

    template <class U>
    constexpr optional& operator=(optional<U>&& rhs);

    template <class U>
    requires(!detail::is_optional<decay_t<U>>)
    constexpr optional& emplace(U&& u) noexcept;

    // \ref{optionalref.swap}, swap
    constexpr void swap(optional& rhs) noexcept;

    // \ref{optional.iterators}, iterator support
    constexpr iterator       begin() noexcept;
    constexpr const_iterator begin() const noexcept;
    constexpr iterator       end() noexcept;
    constexpr const_iterator end() const noexcept;

    // \ref{optionalref.observe}, observers
    constexpr T*       operator->() const noexcept;
    constexpr T&       operator*() const noexcept;
    constexpr explicit operator bool() const noexcept;
    constexpr bool     has_value() const noexcept;
    constexpr T&       value() const;
    template <class U>
    constexpr T value_or(U&& u) const;

    // \ref{optionalref.monadic}, monadic operations
    template <class F>
    constexpr auto and_then(F&& f) const;
    template <class F>
    constexpr auto transform(F&& f) const -> optional<invoke_result_t<F, T&>>;
    template <class F>
    constexpr optional or_else(F&& f) const;

    // \ref{optional.mod}, modifiers
    constexpr void reset() noexcept;

    private:
    T* value_; // exposition only
  };

  template <class T>
  template <class R, class Arg>
  constexpr R optional<T&>::make_reference(Arg&& arg)
  requires is_constructible_v<R, Arg>
  {
    static_assert(std::is_reference_v<R>);
    #if (__cpp_lib_reference_from_temporary >= 202202L)
    static_assert(!std::reference_converts_from_temporary_v<R, Arg>,
    "Reference conversion from temporary not allowed.");
    #endif
    R r(std::forward<Arg>(arg));
    return r;
  }

  //  \rSec3[optionalref.ctor]{Constructors}
  template <class T>
  constexpr optional<T&>::optional() noexcept : value_(nullptr) {}

  template <class T>
  constexpr optional<T&>::optional(nullopt_t) noexcept : value_(nullptr) {}

  template <class T>
  template <class Arg>
  constexpr optional<T&>::optional(in_place_t, Arg&& arg)
  requires is_constructible_v<add_lvalue_reference_t<T>, Arg>
  : value_(addressof(
  make_reference<add_lvalue_reference_t<T>>(std::forward<Arg>(arg)))) {
  }

  template <class T>
  template <class U>
  requires(!detail::is_optional<decay_t<U>>)
  constexpr optional<T&>::optional(U&& u) noexcept : value_(addressof(u)) {
    static_assert(is_constructible_v<add_lvalue_reference_t<T>, U>,
    "Must be able to bind U to T&");
    static_assert(is_lvalue_reference<U>::value, "U must be an lvalue");
  }

  template <class T>
  template <class U>
  constexpr optional<T&>::optional(const optional<U>& rhs) noexcept {
    static_assert(is_constructible_v<add_lvalue_reference_t<T>, U>,
    "Must be able to bind U to T&");
    if (rhs.has_value())
    value_ = to_address(rhs);
    else
    value_ = nullptr;
  }

  // \rSec3[optionalref.assign]{Assignment}
  template <class T>
  constexpr optional<T&>& optional<T&>::operator=(nullopt_t) noexcept {
    value_ = nullptr;
    return *this;
  }

  template <class T>
  template <class U>
  requires(!detail::is_optional<decay_t<U>>)
  constexpr optional<T&>& optional<T&>::operator=(U&& u) {
    static_assert(is_constructible_v<add_lvalue_reference_t<T>, U>,
    "Must be able to bind U to T&");
    static_assert(is_lvalue_reference<U>::value, "U must be an lvalue");
    value_ = addressof(u);
    return *this;
  }

  template <class T>
  template <class U>
  constexpr optional<T&>&
  optional<T&>::operator=(const optional<U>& rhs) noexcept {
    static_assert(is_constructible_v<add_lvalue_reference_t<T>, U>,
    "Must be able to bind U to T&");
    if (rhs.has_value())
    value_ = to_address(rhs);
    else
    value_ = nullptr;
    return *this;
  }

  template <class T>
  template <class U>
  constexpr optional<T&>& optional<T&>::operator=(optional<U>&& rhs) {
    static_assert(is_constructible_v<add_lvalue_reference_t<T>, U>,
    "Must be able to bind U to T&");
    #if (__cpp_lib_reference_from_temporary >= 202202L)
    static_assert(
    !std::reference_converts_from_temporary_v<add_lvalue_reference_t<T>,
    U>,
    "Reference conversion from temporary not allowed.");
    #endif
    if (rhs.has_value())
    value_ = to_address(rhs);
    else
    value_ = nullptr;
    return *this;
  }

  template <class T>
  template <class U>
  requires(!detail::is_optional<decay_t<U>>)
  constexpr optional<T&>& optional<T&>::emplace(U&& u) noexcept {
    return *this = std::forward<U>(u);
  }

  //   \rSec3[optionalref.swap]{Swap}

  template <class T>
  constexpr void optional<T&>::swap(optional<T&>& rhs) noexcept {
    std::swap(value_, rhs.value_);
  }

  // \rSec3[optionalref.iterators]{Iterator Support}
  template <class T>
  constexpr optional<T&>::iterator optional<T&>::begin() noexcept {
    return iterator(has_value() ? value_ : nullptr);
  };

  template <class T>
  constexpr optional<T&>::const_iterator optional<T&>::begin() const noexcept {
    return const_iterator(has_value() ? value_ : nullptr);
  };

  template <class T>
  constexpr optional<T&>::iterator optional<T&>::end() noexcept {
    return begin() + has_value();
  }

  template <class T>
  constexpr optional<T&>::const_iterator optional<T&>::end() const noexcept {
    return begin() + has_value();
  }

  // \rSec3[optionalref.observe]{Observers}
  template <class T>
  constexpr T* optional<T&>::operator->() const noexcept {
    return value_;
  }

  template <class T>
  constexpr T& optional<T&>::operator*() const noexcept {
    return *value_;
  }

  template <class T>
  constexpr optional<T&>::operator bool() const noexcept {
    return value_ != nullptr;
  }
  template <class T>
  constexpr bool optional<T&>::has_value() const noexcept {
    return value_ != nullptr;
  }

  template <class T>
  constexpr T& optional<T&>::value() const {
    if (has_value())
    return *value_;
    throw bad_optional_access();
  }

  template <class T>
  template <class U>
  constexpr T optional<T&>::value_or(U&& u) const {
    static_assert(is_copy_constructible_v<T>, "T must be copy constructible");
    static_assert(is_convertible_v<decltype(u), T>,
    "Must be able to convert u to T");
    return has_value() ? *value_ : std::forward<U>(u);
  }

  //   \rSec3[optionalref.monadic]{Monadic operations}
  template <class T>
  template <class F>
  constexpr auto optional<T&>::and_then(F&& f) const {
    using U = invoke_result_t<F, T&>;
    static_assert(detail::is_optional<U>, "F must return an optional");
    return (has_value()) ? invoke(std::forward<F>(f), *value_)
    : remove_cvref_t<U>();
  }

  template <class T>
  template <class F>
  constexpr auto optional<T&>::transform(F&& f) const
  -> optional<invoke_result_t<F, T&>> {
    using U = invoke_result_t<F, T&>;
    return (has_value()) ? optional<U>{invoke(std::forward<F>(f), *value_)}
    : optional<U>{};
  }

  template <class T>
  template <class F>
  constexpr optional<T&> optional<T&>::or_else(F&& f) const {
    using U = invoke_result_t<F>;
    static_assert(is_same_v<remove_cvref_t<U>, optional>);
    return has_value() ? *value_ : std::forward<F>(f)();
  }

  // \rSec3[optional.mod]{modifiers}
  template <class T>
  constexpr void optional<T&>::reset() noexcept {
    value_ = nullptr;
  }

\end{minted}
\end{document}
